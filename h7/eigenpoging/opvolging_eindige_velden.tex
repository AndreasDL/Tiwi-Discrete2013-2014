%\"

%\newcommand{\Br}[2]{\displaystyle\frac{#1}{#2}}
\documentclass[11pt]{article}
\usepackage{a4wide}
\usepackage[dutch]{babel}
\usepackage{framed}
\usepackage{fancyvrb}
%\usepackage[dvips]{graphicx}
%
\setlength{\parindent}{0pt}
\setlength{\parskip}{\baselineskip}
\pagestyle{plain}
%
\setlength{\textwidth}{18cm}
\setlength{\textheight}{26cm}
%\setlength{\topmargin}{0.5cm}
%\setlength{\headheight}{0.3cm}
%\setlength{\headsep}{0cm}
%\setlength{\oddsidemargin}{0.5cm}
%%\setlength{\marginparsep}{0cm}
%\setlength{\marginparwidth}{0cm}
\addtolength{\hoffset}{-1.5cm}
\addtolength{\voffset}{-2cm}

\begin{document}

\begin{center}
\vspace{-4mm}{Discr Wisk \hfill\bf Eindige velden: vorderingen programmeeroefening \hfill 20 febr 2014}
\end{center}
\hrule
Naam: Andreas De Lille\\
\hrule

\begin{itemize}
\item Welke software / talen gebruik je?
\begin{Verbatim}[frame=single]
Ik heb gekozen voor c++.
\end{Verbatim}

\item Wat moet de gebruiker ingeven indien hij informatie over het veld GF(8) wil verkrijgen? (Antwoord met getallen.)
\begin{Verbatim}[frame=single]
Het programme vraagt een priemgetal en een macht. 
sVoor GF(8) moet ers dus het priemgetal 2 en de macht 3 ingegeven worden.
De verschuivingsfunctie wordt coefficient per coefficient opgevraagd.
\end{Verbatim}

\item Hoe wordt een element van het veld intern bewaard? Gebruik je verschillende voorstellingen voor een element van het veld (of switch je tussen notaties), 
geef dan alle (container-)types waaronder je een element van het veld bewaart. Tip: geef het voorbeeld voor twee elementen uit GF(8). 
\begin{Verbatim}[frame=single]
Een getal wordt intern ogeslagen in de {1 ,1} - notatie.
Dit gebeurd door middel van een vector<int> intern bewaard.
De tabellen zelf worden direct weggeschreven naar stdout.
\end{Verbatim}

\item Wat krijgt de gebruiker als output te zien? Concreet: schrijf de output voor GF(4) volledig uit. (Zie cursusnota's om rekenwerk uit te sparen?)
\begin{Verbatim}[frame=single]
de plustabel en daaronder de maaltabel.

\end{Verbatim}

\item Welke berekeningen / omzettingen heb je allemaal nodig in je programma? Omschrijf bondig en duidelijk. Mag aan de hand van een voorbeeld.
(Antwoord op de achterkant van het blad.)

\begin{Verbatim}[frame=single]
Zie ook code.
- Een functie ontbind die een getal omzet naar de gebruikte notatie.
Dit wordt gedaan door telkens de rest te bereken van de deling met het priemgetal.

- Een functie telOp die 2 vectoren optelt dimensie per dimensie.
Vervolgens houdt hij de rest bij per dimensie.
Deze functie zal de vector aanvullen indien een vector kleiner is dan de andere.

- Een functie plustabel die de plustabel berekent en uitschrijft naar het scherm.
Deze functie zorgt voor de layout en roept telOp aan voor de mogelijke waarden.

- Een functie maal die het product van 2 getallen berekent.
Vervolgens zal deze functie herhaaldelijk de verschuifregel toe passen.
Dit gebeurt eerst door de functie maalAlfa genoeg aan te roepen.
Deze functie zorgt ervoor dat de macht die te groot is wordt omgezet naar 
een macht kleiner. Daarna wordt de originele co\"{e}ffi\"{e}nt mee berekent.
Vervolgens wordt alle op geteld. Per herhaling is de maximum macht van
het resultaat dus smet een vermindert.

Een functie print die een getal wegschrijft naar stdout.

\end{Verbatim}

\end{itemize}
%
\end{document}